\documentclass[11pt,a4paper]{moderncv}

% moderncv themes
%\moderncvtheme[blue]{casual}                 % optional argument are 'blue' (default), 'orange', 'red', 'green', 'grey' and 'roman' (for roman fonts, instead of sans serif fonts)
\moderncvtheme[roman,blue]{classic}                % idem

\usepackage{fontspec,xunicode}
%\setmainfont{Tahoma}
\usepackage[slantfont,boldfont]{xeCJK}
\usepackage{xcolor}                 % replace by the encoding you are using

\usepackage{setspace}
\setCJKmainfont{Adobe Fangsong Std}
%\defaultfontfeatures{Mapping=tex-text}
%\XeTeXlinebreaklocale "zh"
%\XeTeXlinebreakskip = 0pt plus 1pt minus 0.1pt
% moderncv themes

% character encoding


% adjust the page margins
\usepackage[scale=0.84]{geometry}
\setlength{\hintscolumnwidth}{2.3cm}						% if you want to change the width of the column with the dates
\AtBeginDocument{\setlength{\maketitlenamewidth}{4cm}}  % only for the classic theme, if you want to change the width of your name placeholder (to leave more space for your address details
\AtBeginDocument{\recomputelengths}                     % required when changes are made to page layout lengths

% personal data
\firstname{}
\familyname{\textrm{Albert}}
\title{}               % optional, remove the line if not wanted
%\address{xx大学}{310018 浙江杭州}    % optional, remove the line if not wanted
\mobile{TEL: \textbf{150 6815 0000}}                    % optional, remove the line if not wanted
%%\phone{phone (optional)}                      % optional, remove the line if not wanted
%%\fax{fax (optional)}                          % optional, remove the line if not wanted
\email{Email: zhuliting@foxmail.com}                      % optional, remove the line if not wanted
%%\homepage{homepage (optional)}                % optional, remove the line if not wanted
%%\extrainfo{additional information (optional)} % optional, remove the line if not wanted
\photo[60pt]{picture}                         % '64pt' is the height the picture must be resized to and 'picture' is the name of the picture file; optional, remove the line if not wanted
%%\quote{Some quote (optional)}                 % optional, remove the line if not wanted

% to show numerical labels in the bibliography; only useful if you make citations in your resume
%\makeatletter
%\renewcommand*{\bibliographyitemlabel}{\@biblabel{\arabic{enumiv}}}
%\makeatother

% bibliography with mutiple entries
%\usepackage{multibib}
%\newcites{book,misc}{{Books},{Others}}


\nopagenumbers{}                             % uncomment to suppress automatic page numbering for CVs longer than one page


%----------------------------------------------------------------------------------
%            content
%----------------------------------------------------------------------------------
\begin{document}
\maketitle
\linespread{1}
\setlength{\parskip}{0.3\baselineskip}
\section{基本信息}%% 1-2 列
%%\cvline{籍贯}{xx}
\cvcomputer{出生年月}{1986.10} {性别}{男}
\cvcomputer{英语水平}{六级(507)}{籍贯}{xx}
\cvcomputer{学位}{硕士}{研究方向}{xx}
\cvcomputer{联系方式}{\textbf{150 6815 0000}}{Email}{xx@foxmail.com}
\section{教育背景} %% 六列
\cventry{\emph{2009.6-2012.4 }}{xx大学   }{xx专业 }{xx学位}{ }{ }
\cventry{\emph{2005.9-2009.6 }}{xx大学       }{xx专业 }{xx学位}{ }{ }
\section{专业技能}
%\subsection{Miscellaneous}
\cvlistitem{熟悉C,FORTRAN语言,能够熟练编写MPI程序,具备一定的并行程序开发经验.}
\cvlistitem{熟悉Windows和Linux编程环境,对数据结构和算法设计有着深刻理解,有很好的并行计算理论基础.}
\cvlistitem{对虚拟化技术、集群技术有一定的了解.}
\cvlistitem{了解OpenMP,有多线程编程基础和用Shell脚本测试程序的经验.}
\cvlistitem{具备良好的数学基础和数据分析、处理能力,能熟练编写Matlab程序.}
\cvlistitem{具备较强的查阅翻译英文文献能力.能熟练使用Office等办公软件,写过计算机类专利申请书.}
\cvlistitem{能熟练使用Office等办公软件,写过计算机类专利申请书.}
%%\cventry{year--year}{Job title}{Employer}{City}{}{Description line 1\newline{}Description line 2}% arguments 3 to 6 are optional
%%\section{Languages} %%三列
%%\cvlanguage{language 1}{Skill level}{Comment}
\section{获奖情况}
%%\cvcomputer{category 1}{XXX, YYY, ZZZ}{category 4}{XXX, YYY, ZZZ}
\cvlistitem{本科和研究生阶段成绩优秀,先后获优秀学生奖学金七次(本科专业排名前5\%).}
\cvlistitem{本科时期2007年\textbf{高教杯全国大学生数学建模竞赛全国二等奖.}}
\cvlistitem{本科期间获xx大学校"三好学生"、"优良毕业生"等称号.}
\section{实践\&项目经验}
\cventry{\emph{2011.2-2011.9 }}{流体力学数值模拟中的并行化研究}{}{}{}{项目描述: {Linux+MPI+FORTRAN   针对流体力学领域的SIMPLE串行算法,利用区域分解方法对其进行并行化,用FORTRAN语言编写MPI程序,在深腾7000集群上进行实验;对流体力学中求解三对角方程的TDMA算法进行替换,实现一种基于迭代空间交错条块并行的Gauss-Seidel算法,对算法的局部性和通信进行优化.}}% arguments 3 to 6 are optional
\cventry{\emph{2010.8-2010.12}}{稀疏矩阵向量乘法及各种优化技术}{}{}{}{项目描述: {Linux+MPI+C  实现稀疏矩阵向量乘的经典方法,在此基础上进行SIMD、Cache优化、多线程和分布式并行优化,提出一种基于四叉树的矩阵向量乘法并进行相关优化,在深腾1800集群上进行测试.}}
%\renewcommand{\listitemsymbol}{-} % change the symbol for lists
\section{自我评价}
\cvlistitem{自我评价1}
\cvlistitem{自我评价2}
%%\cvlistitem[+]{Item 3}            % optional other symbol
%%\cvlistdoubleitem[\Neutral]{Item 1}{Item 4}
%%\cvlistdoubleitem[\Neutral]{Item 3}{}


\end{document}